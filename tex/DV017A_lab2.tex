% ==============================================================================
%
%                                    DV017A
%                        Inledande Programmering i Java
%                                Laboration #2
%
% Author:   Jonas Sjöberg
%           Högskolan i Gävle
%           tel12jsg@student.hig.se
%           https://github.com/jonasjberg
%
% License:  Creative Commons Attribution-NonCommercial-ShareAlike 4.0
%           International.  See LICENSE.md for full licensing information.
% ==============================================================================

\documentclass[11pt,a4paper]{article}

\usepackage[utf8]{inputenc}
\inputencoding{utf8}
\usepackage[swedish]{babel}
\usepackage[T1]{fontenc}
\usepackage{lmodern}
\usepackage{fullpage}

\usepackage{textcomp}
\usepackage{url}
\usepackage{graphicx}

\usepackage{minted}
%\usemintedstyle{pastie}
\usemintedstyle{bw}

\usepackage{verbatim}
\usepackage{listings}

\usepackage{natbib}

\usepackage[pdfusetitle,bookmarks=true,
 bookmarksnumbered=true,bookmarksopen=false,
 breaklinks=false,pdfborder={0 0 0},backref=false,
 colorlinks=false]{hyperref}

\newmintedfile[javacode]{java}{
%bgcolor=mintedbackground,
%fontfamily=tt,
fontsize=\footnotesize,
linenos=true,
numberblanklines=true,
numbersep=12pt,
numbersep=5pt,
%gobble=0,
frame=lines,
%framerule=0.4pt,
framesep=2mm,
funcnamehighlighting=true,
tabsize=4,
obeytabs=false,
mathescape=false
samepage=false,
showspaces=false,
showtabs =false,
texcl=false,
}

\title{DV017A \\ Java-programmering \\ Laboration 2}

\author{\\
  Jonas Sjöberg\\
  860224\\
  Högskolan i Gävle,\\
  \texttt{tel12jsg@tudent.hig.se}
}

\date{}

\begin{document}
    \maketitle

    \begin{center}
    \begin{tabular}{l r}
        Datum: & \date{}\\
        Kursansvarig lärare: & Atique Ullah
    \end{tabular}
    \end{center}

    \begin{abstract}
    \end{abstract}

    \newpage
    \setcounter{tocdepth}{3}
    \tableofcontents
    \newpage

    \section{Uppgift 1}\label{sec:uppg01}

\subsection{Instruktioner}
\begin{verbatim}
1. Vad är det för speciellt med en konstant variabel och visa hur man
   deklarerar en sådan? Skriv ett litet program på några rader där du använder
   dig av en konstant variabel.
\end{verbatim}


\subsection{Källkod}
\javacode{src/main/Lab2Uppg01.java}
\label{src:uppg01}


\subsection{Skärmdump}
\begin{figure}[htbp]
    \centering
        \includegraphics[width=\linewidth]{img/01.png}
    \caption{Körning av koden till Uppgift~\ref{sec:uppg01}}
\label{fig:uppg01-screenshot}
\end{figure}


    \section{Uppgift 2}\label{sec:uppg02}

\subsection{Instruktioner}
\begin{verbatim}
2. Vilket värde får variablerna x och y efter det att uttrycket (a-g) beräknats
   (varje uttryck ska beräknas enskilt). Lös gärna först uppgiften i huvudet
   (lär man sig mycket på) utan att testa med något program.
   Följande deklaration finns i programmen:

       int x=4;
       int y=2;

       a.  y=x++;
       b.  y=++x;
       c.  y\*=x;
       d.  y+=(x+2);
       e.  y+=--x;
       f.  y+=x++
       g.  x=-y;
\end{verbatim}

\subsection{Lösning}
\subsubsection{Funktion}
% TODO: Eventuell beskrivning av #02 funktionalitet.

\subsubsection{Kommentar}
% NOTERA att (C) använder '\' och inte '/' (division)

%                             int x        int y
% (initial)                       4            2
%       (A)   y    =    x++       5            4
%       (B)   y    =  ++x         6            6 
%       (C)   y  \*=    x         6            6
%       (D)   y   +=   (x + 2)    6            36
%       (E)   y   +=  --x         6            44
%       (F)   y   +=    x++       5            49
%       (G)   x    =   -y;        6            54
%   (final)                      -54           54
%
%
%
%


\subsubsection{Källkod}
\inputminted[linenos]{java}{src/Lab2Uppg02.java}
\caption{Lab2Uppg02.java}
\label{src:uppg02}
% TODO: Lägg till källkod för #02.


\subsubsection{Skärmdump}
\begin{figure}[htbp]
    \centering
        \includegraphics[width=\linewidth]{img/02.png}
    \caption{Körning av koden till Uppgift \ref{sec:uppg02}}
    \label{fig:uppg02-screenshot}
\end{figure}
% TODO: Lägg till skärmdump av #02.


    \section{Uppgift 3}\label{sec:uppg03}

\subsection{Instruktioner}
% TODO: Lägg till instruktioner för #03.


\subsection{Lösning}
\subsubsection{Funktion}
% TODO: Eventuell beskrivning av #03 funktionalitet.

\subsubsection{Kommentar}
% TODO: Eventuell kommentar på #03.


\subsubsection{Källkod}
\inputminted[linenos]{java}{src/Lab2Uppg03.java}
\caption{Lab2Uppg03.java}
\label{src:uppg03}
% TODO: Lägg till källkod för #03.


\subsubsection{Skärmdump}
\begin{figure}[htbp]
    \centering
        \includegraphics[width=\linewidth]{img/03.png}
    \caption{Körning av koden till Uppgift \ref{sec:uppg03}}
    \label{fig:uppg03-screenshot}
\end{figure}
% TODO: Lägg till skärmdump av #03.


    \section{Uppgift 4}\label{sec:uppg#04}

\subsection{Instruktioner}
% TODO: Lägg till instruktioner för #04.


\subsection{Lösning}
\subsubsection{Funktion}
% TODO: Eventuell beskrivning av #04 funktionalitet.

\subsubsection{Kommentar}
% TODO: Eventuell kommentar på #04.


\subsubsection{Källkod}
\inputminted[linenos]{java}{src/Lab2Uppg#04.java}
\caption{Lab2Uppg#04.java}
\label{src:uppg#04}
% TODO: Lägg till källkod för #04.


\subsubsection{Skärmdump}
\begin{figure}[htbp]
    \centering
        \includegraphics[width=\linewidth]{img/#04.png}
    \caption{Körning av koden till Uppgift \ref{sec:uppg#04}}
    \label{fig:uppg#04-screenshot}
\end{figure}
% TODO: Lägg till skärmdump av #04.


    \section{Uppgift 5}\label{sec:uppg05}

\subsection{Instruktioner}
\begin{verbatim}
5. Skriv ett program som räknar förekomsterna av en viss bokstav i en sträng
   (String- objekt), och som sedan skriver ut summan förekomster. Strängen och
   vilken bokstav som ska räknas skrivs in av den som kör programmet. Tips:
   använd metoderna charAt och length i klassen String.
   Exempel på hur utskriften kan se ut:

        Skriv in den sträng du vill leta i: *Kalle har en banan*
        Vilken bokstav vill du räkna: *a*
        Strängen innehåller 4 st a:n.
\end{verbatim}


\subsection{Lösning}
\subsubsection{Funktion}
% TODO: Eventuell beskrivning av #05 funktionalitet.

\subsubsection{Kommentar}
% TODO: Eventuell kommentar på #05.


\subsubsection{Källkod}
\javacode{src/main/Lab2Uppg05.java}
\caption{Lab2Uppg05.java}
\label{src:uppg05}


\subsubsection{Skärmdump}
\begin{figure}[htbp]
    \centering
        \includegraphics[width=\linewidth]{img/05.png}
    \caption{Körning av koden till Uppgift~\ref{sec:uppg05}}
    \label{fig:uppg05-screenshot}
\end{figure}
% TODO: Lägg till skärmdump av #05.


    \section{Uppgift 6}\label{sec:uppg06}

\subsection{Instruktioner}
% TODO: Lägg till instruktioner för #06.


\subsection{Lösning}
\subsubsection{Funktion}
% TODO: Eventuell beskrivning av #06 funktionalitet.

\subsubsection{Kommentar}
% TODO: Eventuell kommentar på #06.


\subsubsection{Källkod}
\inputminted[linenos]{java}{src/Lab2Uppg06.java}
\caption{Lab2Uppg06.java}
\label{src:uppg06}
% TODO: Lägg till källkod för #06.


\subsubsection{Skärmdump}
\begin{figure}[htbp]
    \centering
        \includegraphics[width=\linewidth]{img/06.png}
    \caption{Körning av koden till Uppgift \ref{sec:uppg06}}
    \label{fig:uppg06-screenshot}
\end{figure}
% TODO: Lägg till skärmdump av #06.


    \section{Uppgift 7}\label{sec:uppg07}

\subsection{Instruktioner}
\begin{verbatim}
7. Ett heltal är ett primtal om det bara är delbart med 1 och sig självt.
   Exempelvis så är 2, 3, 5 och 7 primtal, men 4, 6, 8 och 9 är ej primtal.
   Skriv ett program som kontrollerar och skriver ut om ett tal är primtal
   eller ej. Den som kör programmet ska skriva in talet.
   Tips: använd modulus-operatorn %.
\end{verbatim}


\subsection{Lösning}
\subsubsection{Funktion}
% TODO: Eventuell beskrivning av #07 funktionalitet.

\subsubsection{Kommentar}
% TODO: Eventuell kommentar på #07.


\subsubsection{Källkod}
\javacode{src/main/Lab2Uppg07.java}
%\caption{Lab2Uppg07.java}
\label{src:uppg07}


\subsubsection{Skärmdump}
\begin{figure}[htbp]
    \centering
        \includegraphics[width=\linewidth]{img/07.png}
    \caption{Körning av koden till Uppgift~\ref{sec:uppg07}}
    \label{fig:uppg07-screenshot}
\end{figure}
% TODO: Lägg till skärmdump av #07.


    \section{Uppgift 8}\label{sec:uppg#08}

\subsection{Instruktioner}
% TODO: Lägg till instruktioner för #08.


\subsection{Lösning}
\subsubsection{Funktion}
% TODO: Eventuell beskrivning av #08 funktionalitet.

\subsubsection{Kommentar}
% TODO: Eventuell kommentar på #08.


\subsubsection{Källkod}
\inputminted[linenos]{java}{src/Lab2Uppg#08.java}
\caption{Lab2Uppg#08.java}
\label{src:uppg#08}
% TODO: Lägg till källkod för #08.


\subsubsection{Skärmdump}
\begin{figure}[htbp]
    \centering
        \includegraphics[width=\linewidth]{img/#08.png}
    \caption{Körning av koden till Uppgift \ref{sec:uppg#08}}
    \label{fig:uppg#08-screenshot}
\end{figure}
% TODO: Lägg till skärmdump av #08.


    \section{Uppgift 9}\label{sec:uppg09}

\subsection{Instruktioner}
\begin{verbatim}
9. Skriv en klass Artikel som ska representera ett artikelslag på ett
   varulager. Information som ska finnas om varje artikelslag är:

   artikelnr (int), artikelnamn (String), lagerantal (int), pris
   (double)

   Denna information är alltså klassens instansvariabler, passande datatyp står
   inom parentes. I klassen ska även finnas en *klassvariabel* som håller
   räkning på hur många artikelslag det finns totalt, alltså som innehåller
   totala antalet skapade Artikel-objekt:

   totArtiklar (int)

   I klassen ska följande metoder ingå:

   * konstruktorn Artikel, som initierar samtliga instansvariabler samt
     uppdaterar klassvariabelns värde.
   * ändraNamn, metod som via parameter ändrar på artikelnamnet.
   * hamtaNamn, metod som returnerar artikelnamnet.
   * säljaArtikel, metod som minskar lagerantalet med parameterns värde.
   * fyllaLagret, metod som ökar lagerantalet med parameterns värde.
   * ändraPris, metod som ändrar priset till parameterns värde.
   * hamtaPris, metod som returnerar artikelns pris.
   * hamtaTotAntal, klassmetod som returnerar antalet artikelslag.
   * skrivInfo, metod av returtypen void som skriver ut alla data om en artikel.

   Skriv också ett testprogram, där du testar klassens alla metoder.
\end{verbatim}


\subsection{Lösning}
\subsubsection{Funktion}
% TODO: Eventuell beskrivning av #09 funktionalitet.

\subsubsection{Kommentar}
% TODO: Eventuell kommentar på #09.


\subsubsection{Källkod}
\javacode{src/main/Lab2Uppg09.java}
%\caption{Lab2Uppg09.java}
\label{src:uppg09}

\javacode{src/main/Artikel.java}
%\caption{Artikel.java}
\label{src:artikel}


\subsubsection{Skärmdump}
\begin{figure}[htbp]
    \centering
        \includegraphics[width=\linewidth]{img/09.png}
    \caption{Körning av koden till Uppgift \ref{sec:uppg09}}
    \label{fig:uppg09-screenshot}
\end{figure}
% TODO: Lägg till skärmdump av #09.



\end{document}
