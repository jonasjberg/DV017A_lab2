\section{Uppgift 4}\label{sec:uppg04}

\subsection{Instruktioner}
\begin{verbatim}
4. Skriv programmet som spelar det klassiska gissa-talet spelet.  Programmet
   ska först slumpa ut ett tal mellan 1 och 100. Detta tal ska den som spelar
   klura ut. Om man gissat för lågt eller för högt så ska "Du har gissat för
   lågt!" resp "Du har gissat för högt!" skrivas ut på skärmen. Man ska få
   fortsätta gissa tills man gissat rätt, "Rätt gissat!" skrivs då ut. Även
   antal gissningar ska skrivas ut. Varje gång man gissat rätt ska man få välja
   om man vill spela spelet igen eller avsluta.
   Så här kan exempelvis en körning se ut:

        Välkommen till gissa-talet spelet!
        Du ska gissa på ett tal mellan 1 och 100
        gissa\> *75*
        Du har gissat för högt!
        gissa\> *35*
        Du har gissat för lågt!
        gissa\> *60*
        Du har gissat för högt!
        gissa\> *40*
        Rätt gissat! Du har gissat 4 gånger.
        Spela en gång till (j/n): *n*
        Adjö!
\end{verbatim}


\subsection{Lösning}
\subsubsection{Funktion}
% TODO: Eventuell beskrivning av #04 funktionalitet.

\subsubsection{Kommentar}
% TODO: Eventuell kommentar på #04.


\subsubsection{Källkod}
\javacode{src/main/Lab2Uppg04.java}
%\caption{Lab2Uppg04.java}
\label{src:uppg04}


\subsubsection{Skärmdump}
\begin{figure}[htbp]
    \centering
        \includegraphics[width=\linewidth]{img/04.png}
    \caption{Körning av koden till Uppgift~\ref{sec:uppg04}}
    \label{fig:uppg04-screenshot}
\end{figure}
% TODO: Lägg till skärmdump av #04.

