\section{Uppgift 2}\label{sec:uppg02}

\subsection{Instruktioner}
\begin{verbatim}
2. Vilket värde får variablerna x och y efter det att uttrycket (a-g) beräknats
   (varje uttryck ska beräknas enskilt). Lös gärna först uppgiften i huvudet
   (lär man sig mycket på) utan att testa med något program.
   Följande deklaration finns i programmen:

       int x=4;
       int y=2;

       a.  y=x++;
       b.  y=++x;
       c.  y\*=x;
       d.  y+=(x+2);
       e.  y+=--x;
       f.  y+=x++
       g.  x=-y;
\end{verbatim}

\subsection{Lösning}
Lösningen av Uppgift \ref{sec:uppg02} använder en metod
\texttt{shotResults(String label)} för att skriva ut värdet av variablerna
\texttt{x} och \texttt{y} efter att varje uttryck beräknas.

\subsubsection{Funktion}

\subsubsection{Kommentar}

%                                  |  värde efter uttryck  |
%  uttryck  |      operation       |  int x      int y     |
% ==========|======================|=======================|
% (start)   |                      |      4          2     |
%     (A)   |    y    =    x++     |      5          4     |
%     (B)   |    y    =  ++x       |      6          6     |
%     (C)   |    y   *=    x       |      6          36    |
%     (D)   |    y   +=   (x + 2)  |      6          44    |
%     (E)   |    y   +=  --x       |      5          49    |
%     (F)   |    y   +=    x++     |      6          54    |
%     (G)   |    x    =   -y;      |     -54         54    |
%


\subsubsection{Källkod}
\inputminted[linenos]{java}{src/Lab2Uppg02.java}
\caption{Lab2Uppg02.java}
\label{src:uppg02}


\subsubsection{Skärmdump}
\begin{figure}[htbp]
    \centering
        \includegraphics[width=\linewidth]{img/02.png}
    \caption{Körning av koden till Uppgift \ref{sec:uppg02}}
    \label{fig:uppg02-screenshot}
\end{figure}
% TODO: Lägg till skärmdump av #02.

