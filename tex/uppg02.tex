\section{Uppgift 2}\label{sec:uppg02}

\subsection{Instruktioner}
\begin{verbatim}
2. Vilket värde får variablerna x och y efter det att uttrycket (a-g) beräknats
   (varje uttryck ska beräknas enskilt). Lös gärna först uppgiften i huvudet
   (lär man sig mycket på) utan att testa med något program.
   Följande deklaration finns i programmen:

       int x=4;
       int y=2;

       a.  y=x++;
       b.  y=++x;
       c.  y\*=x;
       d.  y+=(x+2);
       e.  y+=--x;
       f.  y+=x++
       g.  x=-y;
\end{verbatim}

\subsection{Lösning}
\subsubsection{Funktion}
% TODO: Eventuell beskrivning av #02 funktionalitet.

\subsubsection{Kommentar}
% NOTERA att (C) använder '\' och inte '/' (division)

%                             int x        int y
% (initial)                       4            2
%       (A)   y    =    x++       5            4
%       (B)   y    =  ++x         6            6 
%       (C)   y  \*=    x         6            6
%       (D)   y   +=   (x + 2)    6            36
%       (E)   y   +=  --x         6            44
%       (F)   y   +=    x++       5            49
%       (G)   x    =   -y;        6            54
%   (final)                      -54           54
%
%
%
%


\subsubsection{Källkod}
\inputminted[linenos]{java}{src/Lab2Uppg02.java}
\caption{Lab2Uppg02.java}
\label{src:uppg02}
% TODO: Lägg till källkod för #02.


\subsubsection{Skärmdump}
\begin{figure}[htbp]
    \centering
        \includegraphics[width=\linewidth]{img/02.png}
    \caption{Körning av koden till Uppgift \ref{sec:uppg02}}
    \label{fig:uppg02-screenshot}
\end{figure}
% TODO: Lägg till skärmdump av #02.

